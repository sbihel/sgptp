\documentclass[a4paper,11pt,english]{article}

\usepackage[T1]{fontenc}
\usepackage[utf8]{inputenc}

\usepackage{babel}

\usepackage{lmodern}
\usepackage{inconsolata} % default typewriter kills me

\usepackage{listings}

\usepackage[htt]{hyphenat} % hyphen with tt

\usepackage{enumitem}
\setlist[enumerate]{leftmargin=*}

\setlength{\parindent}{0cm}

\usepackage{fullpage}

\title{SGP --- TP5\\Scheduling and Synchronization}
\author{Louis Béziaud \and Simon Bihel}

\begin{document}

\maketitle

This report presents how we tested the code that we wrote for the mandatory part of the assignment. It also presents what we have done to add parameters to threads.

\section{Tests}
The program \texttt{test/prodcons.c} is a \textit{multiple producer/consumer} which makes use of \texttt{Semaphore}, \texttt{Lock}, and threads. We mainly ensure two properties. First, we check that a producer (resp.\ consumer) cannot write (resp.\ read) on an unconsumed (resp. unproduced) buffer emplacement. Secondly, we verify that ``nothing is lost'' by maintaining a balance of producing/consuming calls.

Running the program outputs the following code bloc. The absence of ``overflow'' (resp.\ ``underflow''), along with a null balance show that the test went fine.
\begin{lstlisting}[basicstyle=\ttfamily\small,mathescape]
# ./nachos -x prodcons
$\ldots$
>>> balance: 0
>>> num actions: 60
$\ldots$
\end{lstlisting}

Passing parameters to threads was tested though the program \texttt{test/hello.c}. A thread is created from a function which prints each member of its \texttt{argc} arguments.
\begin{lstlisting}[basicstyle=\ttfamily\small,mathescape]
# ./nachos -x hello
$\ldots$
3
Bonjour le monde 
$\ldots$
\end{lstlisting}

We haven't been able to write a good test for \texttt{Conditions} as we can't know for sure if all threads are waiting. Others students have used the fact that a thread can't be interrupted if it is not waiting. That means that if we use interrupt handler we can be sure that all threads are waiting but this is a very specific solution to Nachos and it is quite ugly to poke around manually.



\section{Addition of Parameters at Thread Creation}
To add parameters, we tried different approaches and eventually settled on an
``argc/argv'' solution (we will discuss other solutions later on). Basically,
the \texttt{threadCreate} available to the user has been extended to accept the
two additional parameters. Then \texttt{threadStart} is called and to avoid
problems with additional parameters through syscalls we put all parameters in a
\texttt{struct}. Then, the user's function is called with argc and argv.

Because having to use argc/argv is not natural for user defined functions, we
wanted to explore other solutions. The first step is to make
\texttt{threadCreate} a variadic function (i.e.\ it can accept a variable number
of parameters). Then to pass the parameters to \texttt{threadStart} we have to
make a \texttt{va\_list}, and then again pass it to the user's function. While
the user doesn't have to change the parameters into strings, they have to
manipulate a \texttt{va\_list}. Writing a wrapper to extract parameters from the
\texttt{va\_list} would make things hidden to the user though we haven't found
how to do it. We can't use variadic functions all the way up (or down? hahaha)
because of limited registers to pass arguments.


\end{document}
