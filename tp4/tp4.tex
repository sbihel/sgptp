\documentclass{article}

\usepackage[T1]{fontenc} 
\usepackage[utf8]{inputenc}

\usepackage{fullpage}

\title{SGP -- TP4\\Prise en main de Nachos}
\author{Louis Béziaud \and Simon Bihel}

\begin{document}

\maketitle

\section{Mécanisme d'appel système}

\section{Gestion de threads et de processus}

\begin{enumerate}
\item Lors d'un changement de contexte l'état de l'ancien thread ({\tt SaveProcessorState()}) ainsi que celui du simulateur ({\tt SaveSimulatorState()}) son sauvegardés. {\bf càd ? voir InitThreadContext et InitSimulatorContext}
\item La variable {\tt readyList} est utilisée pour mémoriser les threads prêts à s'exécuter. Le thread actif n'appartient pas à cette liste. Celui-ci est accessible par le pointeur {\tt g\_current\_thread}.
\item La variable {\tt g\_alive} list les threads existants. {\bf différence {\tt g\_alive} {\tt readyList} ?}
\item 
\item Un objet bloqué sur un sémaphore est placé dans la liste {\tt queue} de ce dernier. {\bf où dans le scheduler ?}
\item 
\item
\item
\end{enumerate}

\section{Environnement de développement}

\begin{enumerate}
\item
\item
\item
\end{enumerate}

\end{document}
