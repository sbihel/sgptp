\documentclass{article}

\usepackage[T1]{fontenc} 
\usepackage[utf8]{inputenc}

\usepackage{fullpage}

\title{SGP -- TP4\\Prise en main de Nachos}
\author{Louis Béziaud \and Simon Bihel}

\begin{document}

\maketitle

\section{Mécanisme d'appel système}
Lorsqu'on exécute la commande {\tt ./nachos test/hello} on fourni à la fonction {\tt main} de nachos l'exécutable à exécuter. Cette fonction {\tt main} boot la machine et le système puis crée un processus (pour la mémoire) et un thread (pour l'exécution) pour l'exécutable. Ensuite on fait finir le main thread et quand le dernier thread finira le système sera shutdown {\bf par qui? cpas clair j'ai l'impression que c'est pas encore implémenté}. Quand c'est au tour du thread de notre exécutable, on appel la fonction {\tt n\_printf} qui va calculer la chaine de caractères à afficher et ensuite la passer au driver de la console.

\section{Gestion de threads et de processus}

\begin{enumerate}
\item Lors d'un changement de contexte entre deux threads, il est nécessaire de sauvegarder les contextes utilisateur (avec {\tt SaveProcessorState()}) et noyau (avec {\tt SaveSimulatorState()}). Le contexte du thread est constitué de l'état des registres de la machine MIPS (\texttt{thread\_context.int\_registers} et \texttt{thread\_context.float\_registers}), celui du simulateur regroupe les variables d'état du simulateur (\texttt{simulator\_context.buf}) et le pointeur de pile (\texttt{simulator\_context.stackPointer}).
\item La variable {\tt readyList} est utilisée pour mémoriser les threads prêts à s'exécuter. Le thread actif n'appartient pas à cette liste. Celui-ci est accessible par le pointeur {\tt g\_current\_thread}.
\item La variable {\tt g\_alive} liste les threads existants. {\tt readyList} est donc inclue dans {\tt g\_alive}, en plus des threads actifs, bloqués et terminés.
\item Les routines de gestion de listes n'allouent que leurs propres éléments et ce sont les threads qui se chargent de leurs allocations/désallocations. Seuls les threads peuvent gérer cette tâche puisque qu'ils se mettent eux-mêmes dans certaines listes et ont besoin d'être associés à un processus pour avoir une zone d'adressage.
\item Un thread bloqué sur un sémaphore se trouve dans {\tt g\_alive} et la {\tt queue} du sémaphore en question, mais pas dans {\tt readyList}.
\item Il est possible d'interdire les interruptions à l'aide de \texttt{SetStatus(INTERRUPTS\_OFF)}.
\item {\tt SwitchTo} permet de changer de thread actif. Le changement de contexte entraîne la sauvegarde de l'ancien contexte. Le caractère simulé de Nachos nécessite de sauvegarder deux contextes : celui de l'utilisateur (\texttt{thread\_context}) et celui du noyau (\texttt{simulator\_context}). La méthode \texttt{SaveSimulatorState} (resp. \texttt{RestoreSimulatorState}) sauvegarde (resp. restaure) le contexte du noyau. Les méthodes \texttt{SaveProcessorState} et \texttt{RestoreProcessorState} doivent, respectivement, sauvegarder et restaurer le contexte utilisateur, c'est à dire les variables décrites à la question 2.1.
\item Le champ \texttt{type} permet de vérifier que les objets fournis par l'utilisateur au système sont du bon type. Ce mécanisme permet de repérer un problème de type au niveau du programme (qui entraînerait une lecture mémoire incorrecte) et d'arrêter ce dernier plutôt que l'ensemble du système.
\end{enumerate}

\section{Environnement de développement}

\begin{enumerate}
\item {\tt gdb} est utilisable, il y aussi des statistiques pour les processus.
\item {\tt gdb} permet de suivre l'exécution du programme. Donc typiquement pour des fonction ``not fully implemented'' ça va permettre de voir si des variables ont bien été changées, si le contexte a bien été restauré, {\it etc}.
\item L'exécution des programmes utilisateurs est encadrées par le noyau donc oui (?).
\end{enumerate}

\end{document}
